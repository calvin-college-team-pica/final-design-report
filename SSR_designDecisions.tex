\subsection{Solid State Relay Design Decisions}
The team decided to include actual electric switching device as part of the project for a couple of reasons. The first is that having our own breakers with the monitoring capability built in will be much more user friendly, which is a big goal for the project. All the user has to do is replace their standard electro-mechanical breaker with our breaker and monitor combination device and the change is complete, instead of keeping their breakers and rewiring to accommodate the monitoring device. The second reason is that like the meters, breaker technology for homes has not improved, despite technological advances that could improve electrical safety.

The team looked at a number of different options for the switching component, including SCRs, triacs, FETs, thyristors and solid state relays. The team did not consider electro-mechanical switching technology as it offers no advantages over current electrical safety technology, even with the monitoring option. Of the options, SCRs, triacs, FETs and thyristors all function adequately as a switch, but need supporting circuitry to act as a breaker. Solid state relays, however, do not need additional circuitry and can be implemented in the full design in an out-of-the-box condition. After researching the possibilities, the team decided that the feasibility of designing solid state breakers from scratch was lower than initially anticipated, and decided to focus more on providing a good monitoring system. As a result, the solid state relay was chosen because of its simple design and because it still demonstrates the ability to use solid state technology with our monitoring devices in a user friendly device.

The higher cost was determined to be acceptable for the purposes of prototyping based on the amount of time it would save, but is prohibitive for a final product. A second issue is that solid state relays at the correct current rating also require heat sinks and often forced air cooling. As the team hopes to locate the breaker and monitoring devices in the user's already existing breaker box, where there are no cooling vents, using forced air-cooling is a very difficult option. 
