\section{Design Goals}
\subsection{Provide a physical system that accurately monitors power usage}
Inaccurate information provides no benefit to either the power company or the consumer, despite any added features. Therefore, the system should be as accurate as or more accurate than monitors currently used.

\subsection{Provide manuals for maintenance and general use}
The design team recognizes that no system is perfect and will eventually need maintenance and that most consumers do not have extensive knowledge of electrical systems or components. Providing manuals will assist the consumer in understanding their system and getting the most benefit from it. The design team will look to create a manual that will include an overview of how the system works, how to install and configure the system, and how to maintain the system during the course of normal use. This manual will also contain a troubleshooting guide, support information, liabilities, safety information, and any other pertinent information.

\subsection{Design the system to be modular}
Providing information to both the power company and to the consumer is the main goal of the project, but it is possible that an installation will not include all the subsystems. For example, a consumer may want the consumer-oriented part of the system, while the power company does not want the power-company-oriented subsystem, or vice versa. The modularity goal aims to satisfy all situations without forcing extra costs on any party. To do this, the design team will design the system so that the subsystems providing information to the power company and the subsystems providing information to the consumer do not depend on or require each other.

\subsection{Present power usage information in a way that is understandable to an average consumer}
The goal of the design team is to present the information in an understandable format for the consumer. Because the average consumer does not have an engineering background, the design team would like to display the power usage information in dollars per minute, per hour, per day, per week, per month and per yearly. The system will report these costs in addition to more technical information including voltage, current, power factor and more.

\subsection{Minimize on-site maintenance as much as possible}
In many cases, the consumer calls the power company to fix something as simple as a tripped breaker, and the power companies waste a lot of time just driving to and from the site. The ability to do work remotely allows the power company to minimize this cost. Remotely controlling or monitoring different aspects of the meter also lets the power company quickly assess if a problem still needs attention. Aggravated customers may also become violent towards power-company technicians, so by having remote access, the power company's employees are safer from these threats.