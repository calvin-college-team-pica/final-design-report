\subsection{Target Market Definition} % Fixed
The target market for the entire PICA system comprises both electricity producers and electricity consumers, as set forth by the nature of the subsystems. As the power companies supply and own the electricity meters attached to the buildings to which they supply power, the PICA E-meter appeals only to the market of electricity-producing companies. The other two subsystems, the solid-state breakers and base station, target the power-consuming audience, as the devices will assist in monitoring power flow inside the building, where the power company has no presence. As these two markets are essentially exclusive in both membership and interest in the PICA subsystems, the E-meter will be able to function independently of the other consumer-targeted subsystems, and vice versa.

\subsubsection{Power Companies} % Fixed
As power companies currently distribute the whole-building metering hardware that determines how much energy their customer used, the E-meter clearly targets power companies. In fact, the power companies own the power-measuring hardware external to the buildings to while they provide power, so only they may replace or upgrade those devices. At present, power companies send trained meter-readers to read the data from most traditional power meters under their control. The PICA E-meter subsystem aims to improve on this process by automatically sending the measurements to the power company using a means and protocol selected by the particular company. While this will require some hardware customization for each company, the volume of company-specific production should allow the cost to develop the design to spread into a small per-unit cost.

The PICA E-meter subsystem also provides numerous more measures of power than the simple spinning-dial meters. For example, the E-meter will measure the frequency and the RMS voltage of the incoming supply lines, which help indicate the overall quality of power delivered to the customers in the area. This information may also help diagnose any observed issues with power delivery without dispatching a worker to take measurements by hand. In this way, power companies using the PICA E-meter can improve the quality of the service they provide and can save on the labor costs associated with making a site visit.

\subsubsection{Power Consumers}
Although the power company's customers cannot modify the metering panel installed by the power company, they are free to modify the other power distribution components inside their own buildings. The solid-state breakers fall into this category, and provide previously unavailable measurements regarding power consumption and its location within the building. However, as these breakers will replace the pre-existing breakers inside the building, the consumer must be convinced that using the PICA system is worth the trouble and cost of replacing the mechanical circuit breakers with the more feature-rich PICA breakers. To this effect, the most receptive market for the solid-state breakers includes homeowners and building managers who are curious or concerned about power usage inside their building. That is, the people for whom this information can inspire a meaningful change in practice will likely become the first adopters of the subsystem.

The product may also gain a following as an alternative to mechanical breakers in during the construction of a new home or building. This would likely require that the product already have a proven history of reliability and safety, so the previous group of cost- or environmentally-concerned individuals might have to adopt the produce first. If the PICA solid-state breakers become an alternative during construction, the net cost to the user will be lessened, as the building-to-be will not have any pre-existing breakers to discard or replace.

The base station may apply to either of these two consumer groups, as its primary purpose is to manage and interface with the other systems. It does not specifically require the solid-state breakers or the E-meter, but provides little value in a building without any installed PICA systems. The base station exists solely to manage and collect data from other PICA subsystems, as well as format and display these measurements, so its target audience consists of power consumers whose buildings contain at least one of the E-meter or solid-state breakers.

\subsubsection{Market Coordination}
As the E-meter subsystem caters exclusively to power companies while the other subsystems target power consumers, the complete PICA system has no clearly-defined market; neither of the two markets involved desires the entire system. Despite this separation, a clever marketing strategy could motivate one market to pressure the other.

In one scenario, a power company installs the PICA E-meter onto a select set of their customers' buildings. On its own, this should be a transparent change to the consumer. However, marketing the base station as a way to ``see what the power company sees'' about the power delivered could motivate some of the power-consuming market to purchase a PICA subsystem. Giving the power-consuming market a feeling of empowerment or equality to the power company could therefore motivate more consumer-side subsystem sales.

Conversely, the interests of the power-consuming market could generate interest from the power-producing market. In particular, a consumer who already owns the PICA solid-state breakers may appeal to the power company to provide the PICA E-meter: it would expand the amount of information available and give an accurate sense for the upcoming utility bill. Even without any PICA subsystems, the consumer could prod the power company for a PICA E-meter because it would give the power company more information on power quality, which could in turn increase the quality of service provided to the consumer. In these scenarios, the desires of the power-consumer market could influence demand in the power-producing market.

\subsection{Target Market Research} % Fixed
From the team's visit to Consumer's Energy in Jackson Michigan, power companies are very interested in smart home energy meters, such as PICA provides with the E-meter. The team presented a features list to the representative present. who affirmed that the features now included in the E-meter will be valuable to the power companies. However, power companies are already researching and testing smart meter prototypes and samples, so the E-meter will arrive fairly late relative to its competition. Still, of the thousands of power companies in the United States\cite{EIA_Intro}, the PICA system will surely prove interesting to others who may not have considered alternatives. Fewer than half of the homes expected to receive the smart-meter conversion have been converted to date, so the E-meter is still viable in the power company market\cite{Gtech_Smart_Meters}.

The remainder of the system, the base station and the solid-state breakers, will compete in a market of power consumers. The devices currently face a  market of 6,400,000 customers, according to an estimate from the project's Business 396 companion team. Of these customers, approximately 0.02\% may be interested in PICA, giving 12800 expected sales\cite{Gtech_Renew} given that we are a start up company. Additionally, despite the recent economic downturn, housing spending has experienced an upward trend\cite{Economic_Predictions}, which may indicate a possible increase in the number of houses that could be constructed with the PICA solid-state breaker technology. The market for the consumer-oriented subsystems seems to be healthy, and even growing, despite the recent economic slump.

%\subsubsection{Market Feasibility} % This has been fixed
In order for any product to succeed commercially, its perceived value must meet or exceed the price the customer would pay for it. If the PICA system as a whole is to be a feasible market success, it must surpass its competitors in providing value per price. From a practical standpoint, this involves either selling a comparable product for a lower price than the competition, or producing a superior product at a similar price. The PICA project generally aims to follow the second of these two paths. The solid-state circuit breakers include solid-station circuit breakers and circuit-by-circuit power monitoring, both of which seem to be unusual or even unique features, which in turn means that its feasibility depends on the value of its features rather than a lower price. The base station may be viewed as an accessory to the other subsystems, but its function as the output of the collected information gives it a very high value to anyone who desires the information collected by the other PICA devices, so its feasibility relies on its high perceived value, rather than on undercutting competition. The smart-metering aspects of the E-meter essentially meet the expectations set by other smart meters, so the market feasibility of the E-meter device depends more on the price than on the features, but its ties with the base station can provide additional feature value as well. Overall, the subsystems of the PICA project tend to focus on providing valuable features, rather than on reducing the price below that of the competitors.

\subsection{Consumer Cost Recovery} % this is already fixed
In 2009, the residential monthly electricity bill in the United States averaged to \$104.52\cite{DOE-EIA}. If the PICA base station and solid-state breakers sell with a retail price around \$400, then the investment will amount to approximately four months of electricity bills. If, as is intended, the PICA system allows users to more wisely manage their consumption habits, then homeowners who have purchased the base station and have either the solid-state breakers or the E-meter installed at their house should experience a decrease in their monthly bill, which could recover the initial cost of the system installation. A table summarizing the different savings and payback periods appears as table \ref{tab:cost-recoup}.

\begin{table}[htbp]
 \caption{Table of Cost Recovery Rates} \label{tab:cost-recoup}
 \begin{center}
 \begin{tabular}{|c|c|c|} \hline
 \rowcolor{lightgray}
 Relative Billing Reduced & Average Monthly Savings (\$) & Months to Recover \$400 \\ \hline
 0\% & 0 & - \\ \hline
 0.5\% & 0.52 & 765 \\ \hline
 1\% & 1.05 & 382 \\ \hline
 2\% & 2.09 & 192 \\ \hline
 3\% & 3.14 & 128 \\ \hline
 5\% & 5.23 & 77 \\ \hline
 10\% & 10.45 & 38 \\ \hline
 20\% & 20.90 & 19 \\ \hline
 \end{tabular}
 \end{center}
\end{table}

 As cost reduction rates depend entirely upon the user's response to the information, the recovery period for any given customer cannot be predicted. While reduction rates of $20\%$ and higher may be realized for some individuals, a savings rate of 3\% may be somewhat more typical. Using the \ac{DOE}'s monthly average of 908 kWh consumed per residence, this represents a decrease in average consumption of about 27 kWh per month per household. This amounts to 113 watts saved for eight hours for each of thirty days per month, equivalent to slightly more than one typical incandescent lightbulb. Such a reduction yields a payback period of 128 months, or slightly less than eleven years. If, however, the system could reduce consumption by twice this amount, the payback period would halve to just over five years, or 64 months.

