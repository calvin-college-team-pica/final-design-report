The PICA system addresses several problems with power metering through the use of two metering systems, the E-meter and Smart Breakers, and one data collection center, the base station. The E-meter monitors the line voltage and current, computer power and accumulating the energy used over time. The E-Meter design centers on an MSP430 from Texas Instruments. The Smart Breakers measure circuit-by-circuit information, providing a map of power usage throughout the home or business. Additionaly solid-state relays, part of the Smart Breakers, function as a circuit interruptor in place of the traditional electromechanical breakers. The ADE776 metering chip and ATmega328 microprocessor provide measurement and control respectively for the Smart Breakers. Team PICA chose to prototype the base station using a PC application running on a Linux desktop computer. For a production model the base station would  use a LEON3 soft-processor programmed onto a Spartan V FPGA running  a custom Linux distribution. The team also designed a switching mode power supply to power the E-Meter and the Smart Breakers.

Team PICA developed requirements, test plans, and designed a prototype for each of the four subsystems that would become part of the team's potential production unit. During each of these stages the design team sought to provide the consumer with easy to understand data regarding power usage. The entire design process for Team PICA took place over the course of their senior year at Calvin College in Grand Rapids, MI.