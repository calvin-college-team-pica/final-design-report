% Scheduling section, by Avery
\section{Scheduling}
The team initially created a Gantt chart to establish a timeline for the
project and display the critical path for the tasks in the project. As the
team decided to first focus on the Smart Breakers, they elaborated the
different elements of that subsystem first. Additionally, the team included
the known class deadlines for assignments into the Gantt chart to visualize
how much time to allot to those assignments. These charts are included in
appendix \ref{sec:gantt_charts_appendix}.

In practice, however, the team's usage of the Gantt chart dimished soon
after its introduction. As such, the chart remains essentially the same now
as it was in late October, with many tasks and sub-tasks yet to be filled
and assigned. While the attention paid to the Gantt chart waned, the
emphasis of schedules in team meetings increased. The team would use its
meetings to review the upcoming duedates and assign them to individuals,
much in the same way that the Gantt chart would be used.

Additionally, the team created their own website on Google Sites to provide
a flexible scheduling system independent of Microsoft Project, which is
limited to Engineering computers. After each weekly status update meeting,
during which individuals acquired tasks, Nathan would update the website's
task list to reflect the new assignments and their due dates. While this
system does not allow for the critical-path linkages that Gantt charts
feature, this customizable task list can include any information desired,
such as priority, status, and reviewing information. In some instances, the team mistakenly assigned a small task for the duration of the week, leading to wasted time and letting the project fall behind.

The class deadlines posted on Moodle kept the team focused on some of the long-term goals. Focusing on the larger time scale helped the team allocate more proportionally equal amounts of time for each component of a system. This allowed the team to design by choosing components to fit the system, instead of modifying the system to fit the components.