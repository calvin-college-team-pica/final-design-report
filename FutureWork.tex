\subsection{Future Work}
While the system appears to work well for a prototype, there is still much that needs to be completed before a final product can be released. All of the input networks function properly, the breaker part performs the functions needed, and the team can communicate with the microcontroller. However, the functionality of the ADE7763 has not been verified since the microcontroller could not communicate properly with the chip. The team still needs to test the smart breakers' functionality using non-linear loads, designing communications for several units at the same time, effective solid state technology for switching purposes and selecting/implementing a different metering chip and microcontroller. 

No adjustments should be necessary for non-linear loads, but since large motors often draw a significant amount of current upon startup, the team would like to test to make sure this is not a limitation on the functionality.

Right now, the prototype monitor is set up so that a single monitoring chip reports directly to the computer. However, in a typical home, multiple devices are expected to be used simultaneously, requiring a more complicated process for passing data to ensure no data is lost. Assuming a system with minimal data transmission lines, the team recommends the use of a Master Control Unit to manage the monitoring devices prior to transmitting data back to the base station. Another concern is the bandwidth and ability to handle real-time data from over a dozen devices. If the user desires high precision and frequent readings from many monitoring devices, a single ZigBee system (currently the most likely option) will not be sufficient.

Designing a good solid state breaker to work in the small, un-cooled space provided in a breaker box is another step the team would like to take. However, without changing the design of a typical breaker box, the limited access to cooling and the voltage and current thresholds required make this a very difficult step forward. A new style of breaker box could be provided to allow for better cooling, but this would be completely contradictory to the team's goal of providing a simple and safe installation, requiring a skilled electrician to re-wire the box for the homeowner. 
 
For production purposes, implementing a different metering chip is another critical stage. The ADE7763 from Analog Devices works well as a proof of concept, but is not the most cost effective solution for the consumer. Another possible benefit is that the microcontroller will work better with the chip its designed to work with. This should help utilize more of the available functionality in the metering chips, continuing to provide the customer with more information. 
