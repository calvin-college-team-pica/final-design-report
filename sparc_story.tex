\section{Base Station}
The original design for the base station called for a self-contained Linux
system to collect and process the measurements from the other PICA
systems. To accomplish this goal, the design team decided to use an
\ac{FPGA} development board and program it will the SPARC-compatible LEON3
soft-processor, which is licensed under the \ac{GPL}. While this option
showed a great deal of promise, it ultimately lead to more obstacles than
the team decided could be overcome without detracting from the other areas
of the project. In light of this decision, the role of the base station
will be fulfilled by a standard desktop computer using custom software to
perform the tasks that an embedded system would have performed.


%\begin{figure}[htbp]
%\begin{center}
%\includegraphics[width=0.8\textwidth]{figures/Base_Station_Block_Diagram}
%\caption{Base Station functional block diagram}
%\label{fig:base_station_block_diagram}
%\end{center}
%\end{figure}

\subsection{Device Selection}
As the base station would have collected and formatted the measurements
taken from the other PICA devices, it would have required a fair amount of
processing power to perform its duties. While a modern desktop computer
would provide more than enough computational ability, the design team
focused on creating a single-purpose embedded Linux system that would
perform the necessary tasks without competing with the processes a user
might run on a home computer. In order to build such a device, the design
team required hardware that was flexible enough to allow extra devices such
as XBee radios or large storage devices to attach to it, but would also
have features like networking built-in.

\subsubsection{Board Selection}
In their search for suitable hardware, the design team discovered an
\ac{FPGA} development board that a previous senior design team had ordered
for their own project. This development board, a Digilent ML-509
educational board,  included a 9-pin \ac{RS232} serial port, a standard
networking port, a compact-flash reader, and many configurable pins for
attaching extra devices. In addition to these peripherals, the board
featured a Xilinx Virtex 5 \ac{FPGA}, which could be configured into a wide
variety of different processors or systems for which a standard hardware
description could be found. Upon finding that these features met the
requirements for the base station, the design team selected the board for
making a prototype of the base station.

While the Virtex 5 chip offered flexibility during development, including
it in a production design would drive the price of the base station very
high. As this flexibility is not needed for  production editions of the
base station, a traditional processor would replace the \ac{FPGA} for both
cost and simplicity.

\subsubsection{Processor Selection}
Having selected the development board, the design team looked at several
different options for soft-processors. The team settled on the
SPARC-compatible LEON3 processor, which is licensed under the \ac{GPL}, as
it seemed to provide an outstanding all-in-one package. In addition to the
processor itself, the LEON3 package included interfaces for the development
board's network, display, and serial peripheral controllers. The LEON3
vendor, Gaisler-Aeroflex, provided a host of cross-compilers and tools to
interface with and debug the processor under free or demo licenses. They
also advertise a Linux distribution containing a variety of utilities and
programs. With all these tools available, the design team felt certain that
the LEON3 would provide everything required of the base station.

\subsection{Building the LEON3 Processor}
In order to use the LEON3 soft-processor, the \ac{FPGA} required a
configuration file generated from the LEON3 hardware description files. As
the \ac{FPGA} was a Xilinx part, the design team downloaded ISE, the Xilinx
software suite for creating part-specific configurations from the
description files. Although Xilinx provides a ``Webpack'' edition of ISE
free of charge, its functionality is very limited. In fact, the synthesis
tools for the specific \ac{FPGA} on the board, the XC5VLX110t, were
explicitly disabled without a license for ISE. To work around this issue,
the team requested a thirty-day evaluation license from Xilinx, which
unlocked the required synthesis tools for the processor on the ML-509
board.

\subsubsection{Configuring the Source Files}
The LEON3 hardware description source files came with pre-made
configurations for many different boards, including one for the ML-509. As
such, this configuration could be used directly without requiring further
changes, although the automated process took more than an hour to
complete. When this task completed, it produced a bitfile that configured
the specific model of \ac{FPGA} to behave like the LEON3 processor. The
design team later discovered that Gaisler had a functionally similar
bitfile available for direct download.

\subsubsection{Programming the \ac{FPGA}}
In order to use the Xilinx tool to program the \ac{FPGA}, the workstation
computer needed access to the development board's configuration. Xilinx
provided a programmer pod, a \ac{USB} device to access the \ac{FPGA}'s
configuration using the \ac{JTAG} protocol, for exactly this purpose. Once
fitted with the appropriate drivers, the workstation successfully detected
the pod. The Xilinx program \texttt{impact} wrote the bitfile into the
\ac{FPGA}, transforming it into the LEON3 processor.

\subsubsection{Managing the LEON3 Processor}
Gaisler-Aeroflex provided an evaluation version of their \texttt{grmon}
tool to connect to the LEON3 for loading and debugging code on it. As
described in its manual, \texttt{grmon} could connect to the processor
using the same \ac{JTAG} pod that \texttt{impact} used to program the
device. This claim held true, and \texttt{grmon} promptly gave a summary of
the system it found there.

\subsection{Extending the LEON3}
\typeout{Make sure the article near line \the\inputlineno\ is correct!}
While Gaisler-Aeroflex provided the core LEON3 and its interfaces to
hardware under the \ac{GPL}, they do not distribute an \ac{FPU} under the
same license. As such, on the default LEON3 system, any float-point
operations would be emulated in software, rather than performed in
hardware. As the intended purpose of the base station would likely involve
operations with floating-point numbers, including a hardware \ac{FPU} would
likely benefit the performance of the base station.

While the LEON3 configuration tools include a section for adding an
\ac{FPU} to the system, they did not include the description files for such
a device. Instead, Gaisler provided its own \ac{FPU}s as a separate
download. By separating them from the \ac{GPL}-licensed source, Gaisler
could distribute the \ac{FPU} in a form that could be included into the
LEON3 system without exposing their source description files. Once these
files were added into the project, the LEON3 configuration tool could
include them into the next bitfile. The design team created and loaded the
bitfile onto the \ac{FPGA}, and asked \texttt{grmon} to display information
about the system. Its output is given in the appendix, listing
\ref{lst:grmon_hw}.

\subsection{Running Gaisler Linux on the LEON3 Processor}
As previously mentioned, Gaisler-Aeroflex provided a customized Linux
distribution for the LEON3, which is derived from the SnapGear embedded
Linux distribution. They also provided a pre-built compiling tool chain
based on \ac{GCC}. Using these two together should have produced a small
Linux system for the LEON3. The critical components of the system, such as
the Linux kernel and the LEON3 bootloader, functioned without much extra
tweaking. An example of the messages printed by the Linux kernel over a
serial connection may be found in the appendix, listing
\ref{lst:linux_w_ifconfig}.

Gaisler provided a configuration tool for their distribution that behaved
very similarly to their previous configuration tools. This tool not only
allowed for configuring the kernel, but also for creating an entire Linux
system around it, complete with a shell and a wide range of utilities and
applications. Unfortunately, very few of them worked without
tinkering. Some options could not be built at all, others could be built
with a sufficient amount of tweaking, but even then the configuration files
for these programs did not install correctly into the distribution's
miniature file-system. The design team tried multiple times to get the
system working, but when the distribution's bundled \ac{DHCP} client would
not load due to library linking failures, the team decided to try a
different approach. Message logs with these failures can be found in the
appendix, listings \ref{lst:dhcp_busybox} and \ref{lst:dhcp_bash}.

\subsection{Building Linux from Scratch for the LEON3}
In order to build a Linux system without dealing with Gaisler's
distribution, the design team changed focus to building a Linux
distribution from source code. The team referenced the guide at
\url{http://cross-lfs.org/view/clfs-embedded/arm/}, and adapted it for the
LEON3 SPARC system. While this process ran quite smoothly, it ultimately
failed when trying to boot on the LEON3, where the system unexpectedly
stopped at a break point. This problem persisted even when the debugging
connection asserted that such breakpoints should be ignored. The design
team's modified guide is given in the appendix, section
\ref{sec:sparc_embedded_guide}.

\subsection{Final Decision}
Due to hardware difficulties, the design team believes that the time
required to get a LEON3-powered base station would be better spent on other
areas of the project. As the ZigBee radio receivers work well on desktops,
the team intends to create a monitoring application to run on a
ZigBee-enabled home computer. While this does not allow for an always-on
dedicated device, it provides similar functionality using a device that is
familiar to most home-owners.

