\section{Embedded Linux from Scratch for the LEON3 SPARC}
\label{sec:sparc_embedded_guide}
\lstset{frame=single,language=bash,basicstyle=\small\ttfamily}
 This guide is adapted from \url{http://cross-lfs.org/view/clfs-embedded/arm} to work on the LEON3
 Sparc-V8-compatible hardware. It assumes that the SPARC processor has an \ac{FPU} built into it.


This guide was originally tested and performed on Ubuntu 10.04 LTS. Hence, \texttt{/media} for the base of the
mountpoint directory.


\subsection{Setting up the build environment}
 \subsubsection{Make a new partition}
Use the program ``GParted'' to make an Ext3 partition of 2GB or more, and label it ``CLFS''. Non-Ubuntu
systems will probably have their own particular way of doing this that depends on where disks are mounted.
 \subsubsection{Mount the new partition}
 \begin{itemize}
  \item Mount the partition: see your distribution for details. On Ubuntu, click "Places > CLFS" in the GNOME
    Panel.
  \item Set up an environment variable that points to the partition
\begin{lstlisting}
export CLFS=/media/CLFS
\end{lstlisting}
  \item Create the source directories (as root)
    \begin{lstlisting}
sudo mkdir -v $CLFS/sources
sudo chmod -v a+wt $CLFS/sources
\end{lstlisting}
 \end{itemize}
\subsubsection{Download the CLFS-embedded sources}
 \begin{itemize}
  \item Enter the sources directory
  \begin{lstlisting}
cd $CLFS/sources
  \end{lstlisting}
  \item Make a list of all the sources
  \begin{lstlisting}
cat > embedded_list.txt << EOF
http://ftp.gnu.org/gnu/binutils/binutils-2.21.tar.bz2
http://busybox.net/downloads/busybox-1.17.3.tar.bz2
http://cross-lfs.org/files/packages/embedded-0.0.1/clfs-embedded-bootscripts-1.0-pre5.tar.bz2
http://downloads.sourceforge.net/e2fsprogs/e2fsprogs-1.41.14.tar.gz
ftp://gcc.gnu.org/pub/gcc/releases/gcc-4.5.2/gcc-4.5.2.tar.bz2
http://ftp.gnu.org/gnu/gmp/gmp-5.0.1.tar.bz2
http://cross-lfs.org/files/packages/embedded-0.0.1/iana-etc-2.30.tar.bz2
http://www.kernel.org/pub/linux/kernel/v2.6/linux-2.6.36.3.tar.bz2
http://www.multiprecision.org/mpc/download/mpc-0.8.2.tar.gz
http://gforge.inria.fr/frs/download.php/27105/mpfr-3.0.0.tar.bz2
http://www.uclibc.org/downloads/uClibc-0.9.31.tar.bz2
http://downloads.sourceforge.net/libpng/zlib-1.2.3.tar.gz EOF
  \end{lstlisting}
  \item Download these packages: (\texttt{--no-clobber} lets you skip what you already have)
  \begin{lstlisting}
wget --no-clobber -i embedded_list.txt
  \end{lstlisting}

  \item Make a list of the Leon3-specific files:
  \begin{lstlisting}
cat > leon3_list.txt << EOF
ftp://gaisler.com/gaisler.com/linux/linux-2.6/kernel/mklinuximg-2.6.36-1.0.8.tar.bz2
ftp://gaisler.com/gaisler.com/linux/linux-2.6/kernel/leon-linux-2.6-2.6.36.3-1.0.1.tar.bz2
ftp://gaisler.com/gaisler.com/linux/linux-2.6/kernel/grlib-linux-drvpkg-1.0.0.tar.bz2
EOF
  \end{lstlisting}

  \item Download these files, too.
  \begin{lstlisting}
wget --no-clobber -i leon3_list.txt
  \end{lstlisting}

  \item Make a list of all patches from CLFS:
  \begin{lstlisting}
cat > patch_list.txt << EOF
http://patches.cross-lfs.org/embedded-dev/busybox-1.17.3-config-1.patch
http://patches.cross-lfs.org/embedded-dev/busybox-1.17.3-fixes-1.patch
http://patches.cross-lfs.org/embedded-dev/iana-etc-2.30-update-1.patch
http://patches.cross-lfs.org/embedded-dev/uClibc-0.9.31-configs-2.patch
EOF
  \end{lstlisting}
  \item Download the patches
  \begin{lstlisting}
wget --no-clobber -i patch_list.txt
  \end{lstlisting}
\end{itemize}

 \subsubsection{Add the CLFS user}
 \begin{itemize}
   \item Become root to administer users, and keep CLFS set
\begin{lstlisting}
sudo -s CLFS=$CLFS
\end{lstlisting}
   \item (As root) Create the group and user "clfs"
     \begin{lstlisting}
groupadd clfs
useradd -s /bin/bash -g clfs -m -k /dev/null clfs
passwd clfs
     \end{lstlisting}
  \item (As root) Give the clfs user access to \$CLFS -- Note: \$CLFS should still be set. This should not error.
    \begin{lstlisting}
chown -Rv clfs ${CLFS}
    \end{lstlisting}
 \item Surrender root privileges
\begin{lstlisting}
exit
\end{lstlisting}
 \item Become the clfs user
   \begin{lstlisting}
su - clfs
   \end{lstlisting}
 \end{itemize}
 \subsubsection{Configuring the clfs user}
 \begin{itemize}
  \item Bash profile (when in a login shell)
    \begin{lstlisting}
cat > ~/.bash_profile << "EOF"
exec env -i HOME=${HOME} TERM=${TERM} PS1='\u:\w\$ ' /bin/bash
EOF
    \end{lstlisting}
  \item Bash rc (for non-login shells) 
    \begin{lstlisting}
cat > ~/.bashrc << "EOF"
set +h
umask 022
CLFS=/media/CLFS
LC_ALL=POSIX
PATH=${CLFS}/cross-tools/bin:/bin:/usr/bin
export CLFS LC_ALL PATH
EOF
    \end{lstlisting}
  \item Load the new bash profile
    \begin{lstlisting}
~/.bash_profile
    \end{lstlisting}

 \end{itemize}
 \subsubsection{Preparing the filesystem}
\begin{itemize}
  \item Create directories necessary for Linux to work
\begin{lstlisting}
mkdir -pv ${CLFS}/{bin,boot,dev,{etc/,}opt,home,lib/{firmware,modules},mnt}
mkdir -pv ${CLFS}/{proc,media/{floppy,cdrom},sbin,srv,sys}
mkdir -pv ${CLFS}/var/{lock,log,mail,run,spool}
mkdir -pv ${CLFS}/var/{opt,cache,lib/{misc,locate},local}
install -dv -m 0750 ${CLFS}/root
install -dv -m 1777 ${CLFS}{/var,}/tmp
mkdir -pv ${CLFS}/usr/{,local/}{bin,include,lib,sbin,src}
mkdir -pv ${CLFS}/usr/{,local/}share/{doc,info,locale,man}
mkdir -pv ${CLFS}/usr/{,local/}share/{misc,terminfo,zoneinfo}
mkdir -pv ${CLFS}/usr/{,local/}share/man/man{1,2,3,4,5,6,7,8}
for dir in ${CLFS}/usr{,/local}; do
  ln -sv share/{man,doc,info} ${dir}
done
\end{lstlisting}
  \item Make the cross-tools directories
    \begin{lstlisting}
install -dv $CLFS/cross-tools{,/bin}
    \end{lstlisting}
 \item Set up the necessary files for Linux
 \subitem Mtab symlink (list of mounted file systems)
 \begin{lstlisting}
ln -svf ../proc/mounts ${CLFS}/etc/mtab
\end{lstlisting}
  \subitem Passwd (list of users)
\begin{lstlisting}
cat > ${CLFS}/etc/passwd << "EOF"
root::0:0:root:/root:/bin/ash
daemon:x:2:6:daemon:/sbin:/bin/false
EOF
\end{lstlisting}
  \subitem Group (list of groups)
\begin{lstlisting}
cat > ${CLFS}/etc/group << "EOF"
root:x:0:
bin:x:1:
sys:x:2:
kmem:x:3:
tty:x:4:
tape:x:5:
daemon:x:6:
floppy:x:7:
disk:x:8:
lp:x:9:
dialout:x:10:
audio:x:11:
video:x:12:
utmp:x:13:
usb:x:14:
cdrom:x:15:
EOF
\end{lstlisting}
  \subitem Log files for login, agetty, and init
  \begin{lstlisting}
touch ${CLFS}/var/run/utmp ${CLFS}/var/log/{btmp,lastlog,wtmp}
chmod -v 664 ${CLFS}/var/run/utmp ${CLFS}/var/log/lastlog
  \end{lstlisting}

\end{itemize}


\subsection{Making the Cross-Compile Tools}
 \subsubsection{Setting CFLAGS}
 \begin{lstlisting}
unset CFLAGS
unset CXXFLAGS
echo unset CFLAGS >> ~/.bashrc
echo unset CXXFLAGS >> ~/.bashrc
\end{lstlisting}
 \subsubsection{Setting build settings}
\emph{Note: the CPU setting below was taken from the post at
\url{http://gcc.gnu.org/ml/gcc/2010-11/msg00440.html}}
\begin{itemize}
\item Temporarily
  \begin{lstlisting}
export CLFS_HOST=$(echo ${MACHTYPE} | sed "s/-[^-]*/-cross/")
export CLFS_TARGET="sparc-embedded-linux-gnu"
export CLFS_ARCH="sparc"
export CLFS_CPU="sparchfleonv8"
export CLFS_FLOAT="hard"
  \end{lstlisting}
\item Persistently (you can log out and this will re-load on login)
  \begin{lstlisting}
echo export CLFS_HOST=\""$CLFS_HOST\"" >> ~/.bashrc
echo export CLFS_TARGET=\""$CLFS_TARGET\"" >> ~/.bashrc
echo export CLFS_ARCH=\""$CLFS_ARCH\"" >> ~/.bashrc
echo export CLFS_CPU=\""$CLFS_CPU\"" >> ~/.bashrc
echo export CLFS_FLOAT=\""$CLFS_FLOAT\"" >> ~/.bashrc
  \end{lstlisting}
\end{itemize}
 \subsubsection{Install the Linux headers}
 \begin{itemize}
  \item Unpack
    \begin{lstlisting}
cd $CLFS/sources
tar xf linux-2.6.36.3.tar.bz2
tar xf leon-linux-2.6-2.6.36.3-1.0.1.tar.bz2
tar xf grlib-linux-drvpkg-1.0.0.tar.bz2
    \end{lstlisting}
  \item Apply Leon patches
    \begin{lstlisting}
cd linux-2.6.36.3
for file in `ls ../leon-linux-2.6-2.6.36.3-1.0.1/patches/*.patch`; do
  echo $file; patch -Np1 < $file; sleep 0.5; done
cp -r ../grlib-linux-drvpkg-1.0.0/kernel/drivers .
patch -p1 < drivers/grlib/add_grlib_tree.patch
if test $? -ne 0; then
  echo "obj-\$(CONFIG_SPARC_LEON) += grlib/"; fi >> drivers/Makefile
    \end{lstlisting}
  \item Install headers
    \begin{lstlisting}
make mrproper
make ARCH=${CLFS_ARCH} headers_check
make ARCH=${CLFS_ARCH} INSTALL_HDR_PATH=dest headers_install
cp -rv dest/include/* ${CLFS}/usr/include
    \end{lstlisting}
  \item Cleanup
    \begin{lstlisting}
cd $CLFS/sources
rm -rf linux-2.6.36.3
rm -rf leon-linux-2.6-2.6.36.3-1.0.1
rm -rf grlib-linux-drvpkg-1.0.0
    \end{lstlisting}
 \end{itemize}

 \subsubsection{Install GMP-5.0.1}
 \begin{lstlisting}
cd $CLFS/sources
 \end{lstlisting}
 \begin{itemize}
\item Unpack
  \begin{lstlisting}
tar xf gmp-5.0.1.tar.bz2
  \end{lstlisting}

\item Configure
  \begin{lstlisting}
cd gmp-5.0.1
CPPFLAGS=-fexceptions ./configure --prefix=${CLFS}/cross-tools
  \end{lstlisting}
\item Make, time the result (2m11s for me.)
\begin{lstlisting}
time make 
\end{lstlisting}
\item (Optional) Check that it's correct (2m46s for me)
  \begin{lstlisting}
time make check
  \end{lstlisting}
\item Install
  \begin{lstlisting}
make install
  \end{lstlisting}
\item Cleanup
  \begin{lstlisting}
cd $CLFS/sources
rm -rf gmp-5.0.1
  \end{lstlisting}
 \end{itemize}
 \subsubsection{MPFR-3.0.0}
 \begin{itemize}
  \item Unpack
    \begin{lstlisting}
tar xf mpfr-3.0.0.tar.bz2
cd mpfr-3.0.0
    \end{lstlisting}
  \item Configure
   \begin{lstlisting}
LDFLAGS="-Wl,-rpath,${CLFS}/cross-tools/lib" \
./configure --prefix=${CLFS}/cross-tools --enable-shared \
  --with-gmp=${CLFS}/cross-tools
   \end{lstlisting}
  \item Make and install (1m29s for me)
   \begin{lstlisting}
time { make && make install; }
   \end{lstlisting}
  \item Cleanup
   \begin{lstlisting}
cd $CLFS/sources
rm -rf mpfr-3.0.0
    \end{lstlisting}
 \end{itemize}

 \subsubsection{MPC-0.8.2}
 \begin{itemize}
  \item Unpack 
   \begin{lstlisting}
tar xf mpc-0.8.2.tar.gz
cd mpc-0.8.2
   \end{lstlisting}
  \item Configure
    \begin{lstlisting}
LDFLAGS="-Wl,-rpath,${CLFS}/cross-tools/lib" \
  ./configure --prefix=${CLFS}/cross-tools \
  --with-gmp=${CLFS}/cross-tools \
  --with-mpfr=${CLFS}/cross-tools
    \end{lstlisting}
  \item Build and install (About 21 seconds)
    \begin{lstlisting}
time { make && make install; }
    \end{lstlisting}
  \item Cleanup
    \begin{lstlisting}
cd $CLFS/sources rm -rf mpc-0.8.2
    \end{lstlisting}
 \end{itemize}

\subsubsection{Cross Binutils-2.21}
\begin{itemize}
  \item Extract
    \begin{lstlisting}
cd $CLFS/sources
tar xf binutils-2.21.tar.bz2
    \end{lstlisting}

  \item Make a temporary directory
    \begin{lstlisting}
mkdir binutils-cross 
cd binutils-cross
    \end{lstlisting}

  \item Configure
    \begin{lstlisting}
../binutils-2.21/configure --prefix=${CLFS}/cross-tools \
  --target=${CLFS_TARGET} --with-sysroot=${CLFS} --disable-nls \
  --enable-shared --disable-multilib
    \end{lstlisting}
  \item Build and install (5m11s)
    \begin{lstlisting}
time { make configure-host && make && make install; }
    \end{lstlisting}
  \item Post-install maintenance
    \begin{lstlisting}
cp -v ../binutils-2.21/include/libiberty.h ${CLFS}/usr/include
    \end{lstlisting}
  \item Cleanup
    \begin{lstlisting}
cd $CLFS/sources && rm -rf binutils-2.21 binutils-cross
    \end{lstlisting}
\end{itemize}

\subsubsection{Cross GCC - 4.5.2 -- Static}
 \begin{itemize}
  \item Extract
    \begin{lstlisting}
cd $CLFS/sources tar xf gcc-4.5.2.tar.bz2
    \end{lstlisting}
  \item Make a temporary directory
    \begin{lstlisting}
mkdir -v gcc-cross-static cd gcc-cross-static
    \end{lstlisting}
  \item Configure
    \begin{lstlisting}
AR=ar LDFLAGS="-Wl,-rpath,${CLFS}/cross-tools/lib" \
  ../gcc-4.5.2/configure --prefix=${CLFS}/cross-tools \
  --build=${CLFS_HOST} --host=${CLFS_HOST} --target=${CLFS_TARGET} \
  --with-sysroot=${CLFS} --disable-nls --disable-shared \
  --with-mpfr=${CLFS}/cross-tools --with-gmp=${CLFS}/cross-tools \
  --with-mpc=${CLFS}/cross-tools --without-headers --with-newlib \
  --disable-decimal-float --disable-libgomp --disable-libmudflap \
  --disable-libssp --disable-threads --enable-languages=c \
  --disable-multilib --with-float=${CLFS_FLOAT}
    \end{lstlisting}
  \item  Build (13m4s)
    \begin{lstlisting}
time make all-gcc all-target-libgcc
    \end{lstlisting}
  \item Install (4 seconds) time
    \begin{lstlisting}
make install-gcc install-target-libgcc
    \end{lstlisting}
  \item Cleanup
    \begin{lstlisting}
cd $CLFS/sources && rm -rf gcc-4.5.2 gcc-cross-static
    \end{lstlisting}
\end{itemize}

\subsubsection{uClibc-0.9.31}
\begin{itemize}
  \item Extract
    \begin{lstlisting}
tar xf uClibc-0.9.31.tar.bz2
    \end{lstlisting}
  \item Patch cd uClibc-0.9.31
    \begin{lstlisting}
patch -Np1 -i ../uClibc-0.9.31-configs-2.patch
cp -v clfs/config.arm.big .config
    \end{lstlisting}
  \item Configure
Set arch=sparc (v8 will be selected automatically), and set FPU to be what you have in the Leon (Y/N)
   \begin{lstlisting}
make menuconfig
   \end{lstlisting}
  \item Build (1m39s)
    \begin{lstlisting}
time make
    \end{lstlisting}
  \item Install
    \begin{lstlisting}
make PREFIX=${CLFS} install
    \end{lstlisting}
  \item Cleanup
    \begin{lstlisting}
cd $CLFS/sources && rm -rf uClibc-0.9.31
    \end{lstlisting}
 \end{itemize}

 \subsubsection{GCC-4.5.2 -- Full}
  \begin{itemize}
  \item Extract
    \begin{lstlisting}
cd $CLFS/sources tar xf gcc-4.5.2.tar.bz2
    \end{lstlisting}
  \item Make build directories
   \begin{lstlisting}
mkdir gcc-cross-full cd gcc-cross-full
   \end{lstlisting}
  \item Configure
   \begin{lstlisting}
AR=ar LDFLAGS="-Wl,-rpath,${CLFS}/cross-tools/lib" \
  ../gcc-4.5.2/configure --prefix=${CLFS}/cross-tools \
  --build=${CLFS_HOST} --target=${CLFS_TARGET} --host=${CLFS_HOST} \
  --with-sysroot=${CLFS} --disable-nls --enable-shared \
  --enable-languages=c,c++ --enable-c99 --enable-long-long \
  --with-mpfr=${CLFS}/cross-tools --with-gmp=${CLFS}/cross-tools \
  --with-mpc=${CLFS}/cross-tools --disable-multilib \
  --with-float=${CLFS_FLOAT}
   \end{lstlisting}
  \item Build (20m02s)
   \begin{lstlisting}
time make
   \end{lstlisting}
  \item Install
   \begin{lstlisting}
make install
   \end{lstlisting}
  \item Cleanup
   \begin{lstlisting}
cd $CLFS/sources && rm -rf gcc-cross-full gcc-4.5.2
   \end{lstlisting}
 \end{itemize}


\subsection{Building the System}
\subsubsection{Set up cross-compiling variables for convenience}
\begin{lstlisting}
echo export CC=\""${CLFS_TARGET}-gcc\"" >> ~/.bashrc
echo export CXX=\""${CLFS_TARGET}-gcc\"" >> ~/.bashrc
echo export AR=\""${CLFS_TARGET}-ar\"" >> ~/.bashrc
echo exportAS=\""${CLFS_TARGET}-as\"" >> ~/.bashrc
echo export LD=\""${CLFS_TARGET}-ld\"" >> ~/.bashrc
echo export RANLIB=\""${CLFS_TARGET}-ranlib\"" >> ~/.bashrc
echo export READELF=\""${CLFS_TARGET}-readelf\"" >> ~/.bashrc
echo export STRIP=\""${CLFS_TARGET}-strip\"" >> ~/.bashrc
source ~/.bashrc
\end{lstlisting}

\subsubsection{busybox-1.17.3}
\begin{itemize}\item 
  \item Extract
    \begin{lstlisting}
cd $CLFS/sources tar xf busybox-1.17.3.tar.bz2
    \end{lstlisting}
  \item Patch
    \begin{lstlisting}
cd busybox-1.17.3
patch -Np1 -i ../busybox-1.17.3-fixes-1.patch
patch -Np1 -i ../busybox-1.17.3-config-1.patch
    \end{lstlisting}
  \item Configuration Check
Should not require input, just checks that the configuration is acceptable
\begin{lstlisting}
cp -v clfs/config .config make oldconfig
\end{lstlisting}
  \item Build
    \begin{lstlisting}
make CROSS_COMPILE="${CLFS_TARGET}-"
    \end{lstlisting}
  \item Install
    \begin{lstlisting}
make CROSS_COMPILE="${CLFS_TARGET}-" CONFIG_PREFIX="${CLFS}" install
    \end{lstlisting}
  \item Cleanup
    \begin{lstlisting}
cd $CLFS/sources && rm -rf busybox-1.17.3
    \end{lstlisting}
\end{itemize}

 \subsubsection{e2fsprogs-1.41.14}
  \begin{itemize}
  \item Extract
    \begin{lstlisting}
tar xf e2fsprogs-1.41.14.tar.gz
cd e2fsprogs-1.41.14
    \end{lstlisting}
  \item Make an internal build directory
    \begin{lstlisting}
mkdir -v build
cd build
    \end{lstlisting}
  \item Configure
    \begin{lstlisting}
CC="${CC} -Os" ../configure --prefix=/usr --with-root-prefix="" \
  --host=${CLFS_TARGET} --disable-tls --disable-debugfs \
  --disable-e2initrd-helper --disable-nls \
    \end{lstlisting}
  \item Build (53 seconds)
    \begin{lstlisting}
time make
    \end{lstlisting}
  \item Install
    \begin{lstlisting}
make DESTDIR=${CLFS} install install-libs
    \end{lstlisting}
  \item Cleanup
    \begin{lstlisting}
cd $CLFS/sources && rm -rf e2fsprogs-1.41.14
    \end{lstlisting}
  \end{itemize}
 \subsubsection{iana-etc-2.30}
  \begin{itemize}
  \item Extract
\begin{lstlisting}
tar xf iana-etc-2.30.tar.bz2
cd iana-etc-2.30
\end{lstlisting}
  \item Patch
\begin{lstlisting}
patch -Np1 -i ../iana-etc-2.30-update-1.patch
\end{lstlisting}
  \item Make information up-to-date
\begin{lstlisting}
make get
\end{lstlisting}
  \item Build ($<$1 second)
\begin{lstlisting}
time make
\end{lstlisting}
  \item Install
\begin{lstlisting}
make DESTDIR=${CLFS} install
\end{lstlisting}
  \item Cleanup
\begin{lstlisting}
cd $CLFS/sources && rm -rf iana-etc-2.30
\end{lstlisting}
 \end{itemize}
 \subsubsection{zlib-1.2.3}
 \begin{itemize}
  \item Extract
\begin{lstlisting}
tar xf zlib-1.2.3.tar.gz
cd zlib-1.2.3
\end{lstlisting}
  \item Patch
\begin{lstlisting}
cp configure{,.orig}
sed -e 's/-O3/-Os/g' configure.orig > configure
\end{lstlisting}
  \item Configure
\begin{lstlisting}
./configure --prefix=/usr --shared
\end{lstlisting}
  \item Build ($<$5 seconds)
\begin{lstlisting}
time make
\end{lstlisting}
  \item Install
\begin{lstlisting}
make prefix=${CLFS}/usr install
mv -v ${CLFS}/usr/lib/libz.so.* ${CLFS}/lib
ln -svf ../../lib/libz.so.1 ${CLFS}/usr/lib/libz.so
\end{lstlisting}
  \item Cleanup
\begin{lstlisting}
cd $CLFS/sources && rm -rf zlib-1.2.3
\end{lstlisting}
\end{itemize}
 \subsubsection{Create the fstab file}
\begin{lstlisting}
cat > ${CLFS}/etc/fstab << "EOF"
# Begin /etc/fstab
# This assumes that your CF card has Ext3 on partition 1
#    and swap on partition 2.
# file system mount-point type   options         dump fsck
#                                                     order
/dev/sda1     /           ext3   defaults        1    1
/dev/sda2     swap        swap   pri=1           0    0
proc          /proc       proc   defaults        0    0
sysfs         /sys        sysfs  defaults        0    0
devpts        /dev/pts    devpts gid=4,mode=620  0    0
shm           /dev/shm    tmpfs  defaults        0    0
# End /etc/fstab
EOF
\end{lstlisting}
 \subsubsection{Compile the Linux kernel}
 \begin{itemize}
  \item Extract
\begin{lstlisting}
cd $CLFS/sources
tar xf linux-2.6.36.3.tar.bz2
tar xf leon-linux-2.6-2.6.36.3-1.0.1.tar.bz2
tar xf grlib-linux-drvpkg-1.0.0.tar.bz2
tar xf mklinuximg-2.6.36-1.0.8.tar.bz2
\end{lstlisting}
  \item Apply Leon patches
\begin{lstlisting}
cd linux-2.6.36.3
for file in `ls ../leon-linux-2.6-2.6.36.3-1.0.1/patches/*.patch`; do
  echo $file; patch -Np1 < $file; done
cp -r ../grlib-linux-drvpkg-1.0.0/kernel/drivers .
patch -p1 < drivers/grlib/add_grlib_tree.patch
if test $? -ne 0; then
  echo "obj-\$(CONFIG_SPARC_LEON) += grlib/"; fi >> drivers/Makefile
\end{lstlisting}
  \item Prepare source
\begin{lstlisting}
make mrproper
\end{lstlisting}
  \item Configure -- don't make modules.
\begin{lstlisting}
make ARCH=${CLFS_ARCH} CROSS_COMPILE="${CLFS_TARGET}-" menuconfig
\end{lstlisting}
  \item Build
\begin{lstlisting}
time make ARCH=${CLFS_ARCH} CROSS_COMPILE=${CLFS_TARGET}- zImage
\end{lstlisting}
  \item Make the GRMON image
Look up the MAC address of your device's ethernet port, then substitute it for the 001122334455 below.
\begin{lstlisting}
cd ${CLFS}/sources/mklinuximg-2.6.36-1.0.8
sed -i "s@sparc-linux-@$CLFS_TARGET-@g" mklinuximg
./mklinuximg ../linux-2.6.36.3/arch/${CLFS_ARCH}/boot/image image.dsu \
  -cmdline "console=ttyS0,38400 root=/dev/sda1" -ethmac 001122334455
cp image.dsu ${CLFS}/boot/
cd ../linux-2.6.36.3
\end{lstlisting}
  \item Install the kernel
\begin{lstlisting}
cp arch/${CLFS_ARCH}/boot/uImage ${CLFS}/boot/clfskernel-2.6.36.3
cp System.map ${CLFS}/boot/System.map-2.6.36.3
cp .config ${CLFS}/boot/config-2.6.36.3
\end{lstlisting}
  \item Cleanup
\begin{lstlisting}
cd $CLFS/sources
rm -rf linux-2.6.36.3 leon-linux-2.6-2.6.36.3-1.0.1 grlib-linux-drvpkg-1.0.0
\end{lstlisting}
\end{itemize}
 \subsubsection{Giving control to root}
\begin{itemize}
  \item Become root
\begin{lstlisting}
su -s CLFS=\$CLFS"
\end{lstlisting}
  \item Change the ownership of all of \$CLFS
\begin{lstlisting}
chown -Rv root:root ${CLFS}
\end{lstlisting}
  \item Switch a few to another group
\begin{lstlisting}
chgrp -v utmp ${CLFS}/var/run/utmp ${CLFS}/var/log/lastlog
\end{lstlisting}
  \item Create the two most important nodes
\begin{lstlisting}
mknod -m 0666 ${CLFS}/dev/null c 1 3
mknod -m 0600 ${CLFS}/dev/console c 5 1
\end{lstlisting}
\end{itemize}	
 \subsubsection{CLFS-Bootscripts-1.0-pre5}
 \begin{itemize}
  \item Extract
\begin{lstlisting}
cd ${CLFS}/sources
tar xf clfs-embedded-bootscripts-1.0-pre5.tar.bz2
cd clfs-embedded-bootscripts-1.0-pre5
\end{lstlisting}
  item Install
\begin{lstlisting}
mkdir ${CLFS}/etc/rc.d
mkdir ${CLFS}/etc/rc.d/{init.d,start,stop}
make DESTDIR=${CLFS} install-bootscripts
\end{lstlisting}
  \item Cleanup
\begin{lstlisting}
cd ${CLFS}/sources && rm -rf clfs-embedded-bootscripts-1.0-pre5
\end{lstlisting}
\end{itemize}
 \subsubsection{mdev}
 \begin{itemize}
  \item Fill the file
\begin{lstlisting}
cat > ${CLFS}/etc/mdev.conf << "EOF"
# /etc/mdev/conf
# Symlinks:
# Syntax: %s -> %s

MAKEDEV -> ../sbin/MAKEDEV
/proc/core -> kcore
fd -> /proc/self/fd
mcdx -> mcdx0
radio -> radio0
ram -> ram1
sbpcd -> sbpcd0
sr0 -> scd0
sr1 -> scd1
sr10 -> scd10
sr11 -> scd11
sr12 -> scd12
sr13 -> scd13
sr14 -> scd14
sr15 -> scd15
sr16 -> scd16
sr2 -> scd2
sr3 -> scd3
sr4 -> scd4
sr5 -> scd5
sr6 -> scd6
sr7 -> scd7
sr8 -> scd8
sr9 -> scd9
stderr -> fd/2
stdin -> fd/0
stdout -> fd/1

# Remove these devices, if using a headless system
# You will see an error mdev: Bad line 35
vbi -> vbi0
vcs -> vcs0
vcsa -> vcsa0
video -> video0
# Stop Remove for headless system

# Devices:
# Syntax: %s %d:%d %s
# devices user:group mode

null 0:0 777
zero 0:0 666

urandom 0:0 444

console 0:5 0600
fd0 0:11 0660
hdc 0:6 0660
kmem 0:9 000
mem 0:9 0640
port 0:9 0640
ptmx 0:5 0660
tun[0-9]* 0:0 0640 =net/

sda* 0:6 0660
sdb* 0:6 0660
hda* 0:6 0660
hdb* 0:6 0660

tty 0:5 0660
tty0* 0:5 0660
tty1* 0:5 0660
tty2* 0:5 0660
tty3* 0:5 0660
tty4* 0:5 0660
tty5* 0:5 0660
tty6* 0:5 0660

ttyS* 0:20 640
EOF
\end{lstlisting}
\end{itemize}
 \subsubsection{Profile}
\begin{itemize}
  \item Create the file
\begin{lstlisting}
cat > ${CLFS}/etc/profile << "EOF"
# /etc/profile

# Set the initial path
export PATH=/bin:/usr/bin

if [ `id -u` -eq 0 ] ; then
        PATH=/bin:/sbin:/usr/bin:/usr/sbin
        unset HISTFILE
fi

# Setup some environment variables.
export USER=`id -un`
export LOGNAME=$USER
export HOSTNAME=`/bin/hostname`
export HISTSIZE=1000
export HISTFILESIZE=1000
export PAGER='/bin/more '
export EDITOR='/bin/vi'

# End /etc/profile
EOF
\end{lstlisting}
\end{itemize}
 \subsubsection{Inittab}
\begin{lstlisting}
cat > ${CLFS}/etc/inittab << "EOF"
# /etc/inittab

::sysinit:/etc/rc.d/startup

tty1::respawn:/sbin/getty 38400 tty1
tty2::respawn:/sbin/getty 38400 tty2
tty3::respawn:/sbin/getty 38400 tty3
tty4::respawn:/sbin/getty 38400 tty4
tty5::respawn:/sbin/getty 38400 tty5
tty6::respawn:/sbin/getty 38400 tty6

# Put a getty on the serial line (for a terminal)
# uncomment this line if your using a serial console
#::respawn:/sbin/getty -L ttyS0 115200 vt100

::shutdown:/etc/rc.d/shutdown
::ctrlaltdel:/sbin/reboot
EOF
\end{lstlisting}
 \subsubsection{Hostname}
  \begin{itemize}
  \item Pick a hostname. Set it to \texttt{CLFS\_HOST} (replace localhost below)
\begin{lstlisting}
CLFS_HOST="localhost"
\end{lstlisting}
  \item Make it into a file
\begin{lstlisting}
echo "$CLFS_HOST" > ${CLFS}/etc/HOSTNAME
\end{lstlisting}
 \end{itemize}
 \subsubsection{Hosts file}
  \begin{itemize}
  \item Fill file.
Replace \texttt{CLFS\_HOSTNAME} with your name, as cat  doesn't evaluate it.
\begin{lstlisting}
cat > ${CLFS}/etc/hosts << "EOF"
# Begin /etc/hosts (network card version)

127.0.0.1 localhost
127.0.1.1 CLFS_HOSTNAME

# End /etc/hosts (network card version)
EOF
\end{lstlisting}
\end{itemize}	
  \subsubsection{Network configuration}
\begin{lstlisting}
cat > ${CLFS}/etc/network.conf << "EOF"
# /etc/network.conf
# Global Networking Configuration
# interface configuration is in /etc/network.d/

# set to yes to enable networking
NETWORKING=yes

# set to yes to set default route to gateway
USE_GATEWAY=no

# set to gateway IP address
GATEWAY=192.168.0.1

# Interfaces to add to br0 bridge
# Leave commented to not setup a network bridge
# Substitute br0 for eth0 in the interface.eth0 sample below to bring up br0
# instead
# bcm47xx with vlans:
#BRIDGE_INTERFACES="eth0.0 eth0.1 wlan0"
# Other access point with a wired eth0 and a wireless wlan0 interface:
#BRIDGE_INTERFACES="eth0 wlan0"

EOF

mkdir ${CLFS}/etc/network.d
cat > ${CLFS}/etc/network.d/interface.eth0 << "EOF"
# Network Interface Configuration

# network device name
INTERFACE=eth0

# set to yes to use DHCP instead of the settings below
DHCP=no

# IP address
IPADDRESS=192.168.1.2

# netmask
NETMASK=255.255.255.0

# broadcast address
BROADCAST=192.168.1.255
EOF

cat > ${CLFS}/etc/udhcpc.conf << "EOF"
#!/bin/sh
# udhcpc Interface Configuration
# Based on http://lists.debian.org/debian-boot/2002/11/msg00500.html
# udhcpc script edited by Tim Riker <Tim@Rikers.org>

[ -z "$1" ] && echo "Error: should be called from udhcpc" && exit 1

RESOLV_CONF="/etc/resolv.conf"
RESOLV_BAK="/etc/resolv.bak"

[ -n "$broadcast" ] && BROADCAST="broadcast $broadcast"
[ -n "$subnet" ] && NETMASK="netmask $subnet"

case "$1" in
        deconfig)
                if [ -f "$RESOLV_BAK" ]; then
                        mv "$RESOLV_BAK" "$RESOLV_CONF"
                fi
                /sbin/ifconfig $interface 0.0.0.0
                ;;

        renew|bound)
                /sbin/ifconfig $interface $ip $BROADCAST $NETMASK

                if [ -n "$router" ] ; then
                        while route del default gw 0.0.0.0 dev $interface ; do
                                true
                        done

                        for i in $router ; do
                                route add default gw $i dev $interface
                        done
                fi

                if [ ! -f "$RESOLV_BAK" ] && [ -f "$RESOLV_CONF" ]; then
                        mv "$RESOLV_CONF" "$RESOLV_BAK"
                fi

                echo -n > $RESOLV_CONF
                [ -n "$domain" ] && echo search $domain >> $RESOLV_CONF
                for i in $dns ; do
                        echo nameserver $i >> $RESOLV_CONF
                done
                ;;
esac

exit 0
EOF

chmod +x ${CLFS}/etc/udhcpc.conf
cat > ${CLFS}/etc/resolv.conf << "EOF"
# Begin /etc/resolv.conf

domain example.org
nameserver 192.168.0.1
nameserver 192.168.1.1

# End /etc/resolv.conf
EOF
\end{lstlisting}
\subsection{Finishing Up}
 \subsubsection{Make a final directory}
  \begin{itemize}
  \item Copy to a new directory
\begin{lstlisting}
install -dv ${CLFS}-final
cp -arv ${CLFS}/* ${CLFS}-final/
\end{lstlisting}
  \item Remove the unnecessary things
\begin{lstlisting}
rm -rfv ${CLFS}-final/cross-tools
rm -rfv ${CLFS}-final/usr/src/*
rm -rfv ${CLFS}-final/usr/include
rm -rfv ${CLFS}-final/usr/man
rm -rfv ${CLFS}-final/usr/share/man
rm -rfv ${CLFS}-final/sources
FILES="$(ls ${CLFS}-final/lib/*.a ${CLFS}-final/usr/lib/*.a)"
for file in $FILES; do rm -fv $file; done
\end{lstlisting}
 \end{itemize}
 \subsubsection{Create a tarball}
\begin{lstlisting}
install -dv ${CLFS}/build
cd ${CLFS}-final
tar jcfv ${CLFS}/build/clfs-leon3-embedded.tar.bz2 *
\end{lstlisting}

